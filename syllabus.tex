\documentclass{article}

\usepackage[top=1in,left=1in,right=1in,bottom=1in]{geometry}
\usepackage{verbatim}

\begin{document}

\title{CSE 482/682: Artificial Intelligence}
\author{Richard Kelley}
\date{Fall 2014}
\maketitle

\section*{Instructor information}

\begin{itemize}
  \item Your instructor is \textbf{Richard Kelley}
  \item His email address is \verb|richard.kelley@gmail.com|
  \item Office hours by appointment, unless we come up with something better.
\end{itemize} 

\section*{Course description and prerequisites}

\textbf{Official Course Description:} Problem solving, search, and
game trees. Knowledge representation, inference, and rule-based
systems. Semantic networks, frames, and planning. Introduction to
machine learning, neural-nets, and genetic algorithms. (Formerly CS
476; implemented Spring 2005.)\\

\noindent
\textbf{Prerequisite(s):} CS 365.\\

\noindent
\textbf{Richard's Description:} This course will cover the fundamental
principles that allow intelligent systems to operate in the world. We
will look at classical (but still relevant) approaches to artificial
intelligence and then move on the ideas (probabilistic modeling and
learning) that have enabled AI to become an essential and ubiquitous
part of modern civilization.

\section*{List of required course materials}

\textbf{Required Textbook: } \textit{Artificial Intelligence: A Modern
  Approach} by Stuart Russell and Peter Norvig. The textbook website is
\verb|http://aima.cs.berkeley.edu/|. This is actually one of the best
textbooks in Computer Science. It gives a great introduction to the
field, and is a great reference.

\section*{Topics outline} 

The topics below correspond sections in Russell and Norvig, but you
should only consider them a guide to what we'll go over. Broadly, the
three areas we're going to look at are classical AI, the modeling of
uncertainty, and machine learning.\\

\noindent
Tentatively, here's what we're going to cover:

\begin{itemize}
  \item \textbf{Introduction}
    \begin{itemize}
      \item What is AI?
    \end{itemize}

  \item \textbf{Intelligent Agents}
    \begin{itemize}
      \item Agents and Environments
      \item Good Behavior: The Concept of Rationality
      \item The Nature of Environments
      \item The Structure of Agents
    \end{itemize}

  \item \textbf{Solving Problems by Searching}
    \begin{itemize}
      \item Problem-Solving Agents
      \item Example Problems
      \item Searching for Solutions
      \item Uninformed Search Strategies
      \item Informed (Heuristic) Search Strategies
      \item Heuristic Functions
    \end{itemize}

  \item \textbf{Beyond Classical Search}
    \begin{itemize}
      \item Local Search Algorithms and Optimization Problems
      \item Local Search in Continuous Spaces
      \item Online Search Agents and Unknown Environments
    \end{itemize}

  \item \textbf{Quantifying Uncertainty}
    \begin{itemize}
      \item Acting under Uncertainty
      \item Basic Probability Notation
      \item Inference Using Full Joint Distribution
      \item Independence
      \item Bayes' Rule and Its Use
    \end{itemize}

  \item \textbf{Probabilistic Reasoning}
    \begin{itemize}
      \item Representing Knowledge in an Uncertain Domain
      \item The Semantics of Bayesian Networks
      \item Efficient Representation of Conditional Distributions
      \item Exact Inference in Bayesian Networks
      \item Approximate Inference in Bayesian Networks
    \end{itemize}

  \item \textbf{Probabilistic Reasoning Over Time}
    \begin{itemize}
      \item Time and Uncertainty
      \item Inference in Temporal Models
      \item Hidden Markov Models
      \item Kalman Filters
    \end{itemize}

  \item \textbf{Learning from Examples}
    \begin{itemize}
      \item Forms of Learning
      \item Supervised Learning
      \item The Theory of Learning
      \item Regression and Classification with Linear Models
      \item Artificial Neural Networks
      \item Nonparametric Models
      \item Support Vector Machines
    \end{itemize}

  \item \textbf{Applications}
    \begin{itemize}
      \item Natural Language Processing
      \item Robotics
      \item Computer Vision
    \end{itemize}

  \item \textbf{The Future of AI}
\end{itemize}

\section*{Approximate schedule of exams}

There will be three exams and a comprehensive final. The first exam
will cover classical AI (Introduction, Intelligent Agents, Solving
Problems by Searching, Beyond Classical Search). The second exam will
cover probabilistic reasoning (quantifying uncertainty, probabilistic
reasoning, probabilistic reasoning over time). The third exam will
cover machine learning and applications. The final exam will be
comprehensive. The exact dates will depend on our progress through the
material.\\

\noindent
There will also be weekly quizzes, on Tuesdays.

\section*{Grading}

Your grade will be determined by your performance on the quizzes and
exams. The percentage breakdown is:

\begin{itemize}
  \item 20\% quizzes
  \item 10\% exam 1
  \item 20\% exam 2
  \item 20\% exam 3
  \item 30\% final exam
\end{itemize}

Additionally, graduate students will be required to complete a small
(but nontrivial) project related to their intended area of
research. They will have to present their work before the end of the
semester. The items listed above will constitute 70\% of the graduate
section grade. The project will constitute 30\%.

\section*{Statement on Academic Dishonesty}
Cheating, plagiarism or otherwise obtaining grades under false
pretenses constitute academic dishonesty according to the code of this
university. Academic dishonesty will not be tolerated and penalties
can include canceling a student's enrollment without a grade, giving
an F for the course or for the assignment. For more details, see the
University of Nevada, Reno General Catalog.

\section*{Statement of Disability Services}
Any student with a disability needing academic adjustments or
accommodations is requested to speak with the Disability Resource
Center (Thompson Building, Suite 101) as soon as possible to arrange
for appropriate accommodations.

\section*{Statement on Audio and Video Recording}
Surreptitious or covert video-taping of class or unauthorized audio
recording of class is prohibited by law and by Board of Regents
policy. This class may be videotaped or audio recorded only with the
written permission of the instructor. In order to accommodate students
with disabilities, some students may be given permission to record
class lectures and discussions. Therefore, students should understand
that their comments during class may be recorded.

\section*{Statement for Academic Success Services}
Your student fees cover usage of the Math Center (775) 784-4422,
Tutoring Center (775) 784-6801, and University Writing Center (775)
784-6030. These centers support your classroom learning; it is your
responsibility to take advantage of their services. Keep in mind that
seeking help outside of class is the sign of a responsible and
successful student.

\end{document}
